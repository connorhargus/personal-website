\documentclass[11pt]{article}

\usepackage{amsmath,amssymb,amsfonts}
\usepackage{graphicx}
\usepackage{pgfplots}
\usepackage{multicol}
\usepackage{enumitem}
\usepgfplotslibrary{fillbetween}
\pgfplotsset{compat=1.16,width=10cm}


\setlength{\topmargin}{-.5in} \setlength{\textheight}{9.25in}
\setlength{\oddsidemargin}{0in} \setlength{\textwidth}{6.8in}


\begin{document}

\Large

\noindent{\bf Name: \hfill Date: \hfill Quiz 4 \hfill Precalculus - Hargus}

\medskip\hrule
\vspace{10pt}

\noindent \textbf{Instructions:} Please \textbf{show all work} (partial credit will be given for correct work, even if your answer is wrong).


\begin{enumerate}

\item (10 points) Simplify the expression to either 1 or -1.

\begin{enumerate}[itemsep=60pt, label={\alph*)}]
    \item $\displaystyle \sin(-x) \csc(-x) $
    \item $\displaystyle -(\cos^2(x) + \sin^2(x)) $
\end{enumerate}
\vspace{60pt}

\item (10 points) Prove the identity.
\begin{multicols}{2}
\begin{enumerate}[itemsep=60pt, label={\alph*)}]
    \item $\displaystyle \sin(x) = \frac{\tan(x)}{\sec(x)}$
    \item $\displaystyle \frac{\cos(x)}{1 - \cos^2(x)} = \cot(x) \csc(x)$
\end{enumerate}
\end{multicols}

\newpage

\item (10 points) Use a sum or difference identity to find the exact value of $\cos(15^{\circ})$
\vspace{100pt}
\begin{flushright}
$\cos(15^{\circ})= \rule{3cm}{0.4pt}$
\end{flushright}
\vspace{20pt}

\item (10 points) Find all solutions to the equation $\sin(2x) = \sin(x)$ in the interval $[0,2\pi )$.
\vspace{100pt}
\begin{flushright}
$x= \rule{3cm}{0.4pt}$
\end{flushright}
\vspace{20pt}


\item (15 points) Find an \textbf{explicit} rule for the $n$th term of the sequence.
\begin{enumerate}[itemsep=30pt, label={\alph*)}]
\item 1, 5, 9, 13, ...
\begin{flushright}
$a_n$= $\rule{3cm}{0.4pt}$
\end{flushright}
\item $-\frac{1}{2}$, 1, -2, 4, ...
\begin{flushright}
$a_n$= $\rule{3cm}{0.4pt}$
\end{flushright}
\item $a_1 = 5$, $a_n$ = $a_{n+1} +2$
\begin{flushright}
$a_n$= $\rule{3cm}{0.4pt}$
\end{flushright}
\end{enumerate}

\newpage

\item (15 points) You do not need to simplify your answers for these questions, answers with powers, products, and factorials are okay.
\begin{enumerate}[itemsep=60pt, label={\alph*)}]
\item How many ways are there make a license plate with any 3 digits (10 options) and then any 3 letters (26 options)? For instance, one license plate would be 357AYB.
\begin{flushright}
Ways: $\rule{3cm}{0.4pt}$
\end{flushright}
\item How many ways are there to select a group of 3 students from a class of 9 students?
\begin{flushright}
Ways: $\rule{3cm}{0.4pt}$
\end{flushright}
\item How many ways are there to rearrange the letters in the name JIMMY? (for instance, MYJIM is one way)
\begin{flushright}
Ways: $\rule{3cm}{0.4pt}$
\end{flushright}
\item If we flip a coin 5 times, what is the probability that we get the sequence HTHHT in that order?
\begin{flushright}
Probability: $\rule{3cm}{0.4pt}$
\end{flushright}
\item If we flip a coin 5 times, what is the probability that we get heads exactly 3 times if the order doesn't matter?
\begin{flushright}
Probability: $\rule{3cm}{0.4pt}$
\end{flushright}
\end{enumerate}

\newpage


\item (5 points) Compute the sum of the arithmetic series where $a_n = 4n + 2$ for the first 100 terms.

\vspace{60pt}


\item (12 points) True or false? (circle your answer)
\begin{enumerate}[itemsep=15pt, label={\alph*)}]
\item With two six-sided dice, the chance of rolling a 4 is the same as the chance of rolling a 6.\\ \null\hfill \textbf{T} or \textbf{F}
\item A term in an arithmetic sequence is the last term plus some constant.  \\ \null\hfill \textbf{T} or \textbf{F}
\item The sequence $3, 7,  11, 15...$ converges. \null\hfill \textbf{T} or \textbf{F}
\item The series $\sum_{k=1}^{\infty} (\frac{-3}{4})^k$ converges. \null\hfill \textbf{T} or \textbf{F}

\end{enumerate}

\vspace{20pt}

\item (\textbf{Extra Credit:} 5 points) Prove the following statement for all positive integers $n$ using induction. \\
$$8 + 10 + 12 + ... + (2n + 6) = n^2 + 7n$$

\end{enumerate}

\end{document} 